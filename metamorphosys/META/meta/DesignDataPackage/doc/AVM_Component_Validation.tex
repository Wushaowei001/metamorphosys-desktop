
\chapter{Component Model Comformance Suite}
A "conformance suite" is also provided for checking the validity of AVM Component Models. In addition to checking ACM files against the XML schema, the structures within are checked for semantic consistency and well-formedness.

\section{Schema}
An XML Validator included in the libxml package of Python is used to validate the submitted ACM (xml) file against AVM Component Model schema.  
See \textit{\autoref{ACMSchema}: \nameref{ACMSchema}}.

\section{Python Validator Script}
The Python validator script (drop\_test.py) parses the AVM Component spec model and performs the following semantic checks on the component model.

\begin{itemize}
\item Check Resources
\item Check Properties and Values
\item Check Connectors
\item Check Domain Models
\subitem Check CAD Domain Models
\subsubitem Check Datums
\subsubitem Check Parameters
\subitem Check Modelica Domain Models
\subsubitem Check Connector Definitions
\subsubitem Check Medium Specifications
\subsubitem Check Parameters
\subitem Check Manufacturing Domain Models
\subsubitem Check Parameters
\end{itemize}

The listing of the Python Validator Script is provided below:

\begin{MyVerbatim}
from os import listdir
from os.path import isfile, join
from optparse import OptionParser
from fnmatch import fnmatch
from glob import glob
from lxml import etree
import sys
import tablib

XSI = "{http://www.w3.org/2001/XMLSchema-instance}"

# Stats dictionary structure
# comp_name, file_name, schema_valid, has_cad, has_modelica, 
      has_manufacturing, valid_resources,
#              num_connectors, num_properties

# Category component counts
catCounts = dict()

def list_cxmls(mypath):  # return a list of xml files in the given directory
    """Return a list containing component xml files in a given directory."""
    xmls = [join(mypath,f) for f in listdir(mypath) if 
        isfile(join(mypath,f)) and fnmatch(f, '*.acm')]
    return xmls


def schema_validate(xml, xsd):  # validates provided xml files against schema
    """Return true or false based on whether provided XML is valid against 
       schema"""
    xmlschema_doc = etree.parse(xsd)
    xmlschema = etree.XMLSchema(xmlschema_doc)
    doc = etree.parse(xml)
    if xmlschema.validate(doc):
        return doc
    else:
        return None


def check_classification(xmldoc, statsData):  # checks that the component 
                                                has a valid classification tag
    root = xmldoc.getroot()
    classif = root.findall('Classifications')
    hasClassif = True
    classifs = ''
    if classif is None or len(classif) == 0:
        hasClassif = False
    else:
        classifs = classif[0].text
    statsData.append(classifs)
    if classifs in catCounts:
        catCounts[classifs] += 1
    else:
        catCounts[classifs] = 1
    return hasClassif


def check_resources(xmldoc, mypath, statsData):  # checks that the resource 
                                                   link in the XML file are valid
    """Return true or false based on whether the resource objects in the 
       XMl point to valid files"""
    root = xmldoc.getroot()
    resources = root.findall("ResourceDependency")
    validResources = True
    for child in resources:
        res = child.get("Path")
        res_w = join(mypath, res) + '*'
        if len([n for n in glob(res_w) if isfile(n)]) == 0:
            validResources = False
    statsData.append(validResources)
    return True

def check_domain_resources(domain, root):
    uses = domain.get("UsesResource")
    resources = root.findall("ResourceDependency")
    hasValidResource = False
    for res in resources:
        rid = res.get("ID")
        if uses in rid:
            hasValidResource = True
            break
    if not hasValidResource:
        print ' '


    return True


def check_cad_model(cadroot, root):
    datums = cadroot.findall("Datum")
    hasCoordinateSystem = False
    hasPlanesAxis = False
    for d in datums:
        dtype = d.get(XSI + "type")
        if "CoordinateSystem" in dtype:
            hasCoordinateSystem = True
        elif "Point" in dtype or "Axis" in dtype or "Plane" in dtype:
            hasPlanesAxis = True

    if not hasPlanesAxis:
        print ' '

    return True


def check_modelica_model(moroot, root):
    return True


def check_manufacturing_model(maroot, root):
    return True


def check_domain_models(xmldoc, statsData):  # checks that the resource link in the XML file are valid
    """Return true or false based on whether the resource objects in the 
       XMl point to valid files
    :param xmldoc: the XML element tree
    """
    root = xmldoc.getroot()
    domains = root.findall("DomainModel")
    hasCAD = False
    hasModelica = False
    hasManufacturing = False

    for child in domains:
        res = child.get(XSI + "type")
        check_domain_resources(child, root)
        if "CADModel" in res:
            hasCAD = check_cad_model(child, root)
        elif "ModelicaModel" in res:
            hasModelica = check_modelica_model(child, root)
        elif "ManufacturingModel" in res:
            hasManufacturing =  check_manufacturing_model(child, root)

    if not hasManufacturing:
        print 'ERROR: Component[{0}]: No Manufacturing Model'.format(root.get('Name'))
    statsData.extend([hasCAD, hasModelica, hasManufacturing])

    return True


def is_number(s):
    if s is not None:
        try:
            float(s)
            return True
        except ValueError:
            return False
    else: return False


def is_list(myinput):
    par=myinput.replace('[','')
    par=par.replace(']','')
    mat=par.split(',')
    p = True 
    for x in mat:
        if not is_number(x):p=False
    if p: return True
    else: return False


def check_property(prop, proptype, root, propval, propnames2, errorcheck, 
       warningcheck, fileoutput, concheck):  # checks a specific property
    if "CompoundProperty" in proptype:
        # check if compound is non empty and has unique children
        pprops = prop.findall("PrimitiveProperty")
        cprops = prop.findall("CompoundProperty")
        props = pprops + cprops
        if props is None or len(props) == 0:
            errorcheck += 1
            fileoutput.write('WARNING: Component[{0}]: Compound 
              Property[{1}]: has no child property\n'.\
                format(root.get('Name'), prop.get('Name')))
        propnames = [p.get('Name') for p in props]
        if len(propnames) != len(set(propnames)):
            errorcheck += 1
            fileoutput.write('ERROR: Component[{0}]: Compound 
              Property[{1}]: duplicate child properties\n'.\
                format(root.get('Name'), prop.get('Name')))
        for p in cprops:
            notes = check_property(p, 'CompoundProperty', root, propval, 
              propnames2, errorcheck, warningcheck, fileoutput, concheck)
            warningcheck = notes[0];
            errorcheck = notes[1];
            propval = notes[2];
            propnames2 = notes[3];
            concheck = notes[4];
        for p in pprops:
            notes = check_property(p, 'PrimitiveProperty', root, propval, 
              propnames2, errorcheck, warningcheck, fileoutput, concheck)
            warningcheck = notes[0];
            errorcheck = notes[1];
            propval = notes[2];
            propnames2 = notes[3];
            concheck = notes[4];
    elif "PrimitiveProperty" in proptype:
        # check if primitive has a unique value
        vals = prop.findall('Value')
        if vals is None or len(vals) != 1:
            errorcheck += 1
            fileoutput.write('ERROR: Component[{0}]: Primitive 
             Property[{1}]: has illegal value\n'.\
                format(root.get('Name'), prop.get('Name')))

            ## print prop.get('Name')
        else:
            ##if len(propnames2) < 16:
                ValEx=vals[0].findall('ValueExpression')
                basicAtt1=vals[0].attrib.get('DataType')
                basicAtt2=vals[0].attrib.get('Dimensions')
                basicAtt3=vals[0].attrib.get('Unit')
                basicAtt4=ValEx[0].attrib.get('ValueSource')
                
##                print ValEx
                if (basicAtt4 != None):
##                    print basicAtt4
                    if ('CAD' in basicAtt4 or 'cad' in basicAtt4):
                        concheck += 1;
##                        print concheck
##                    root = basicAtt4.getroot()
##                    print root
                    
                if ValEx != []:
                    aa=ValEx[0].findall('Value')
                    if aa != []:
                        myVal=aa[0].text
                        if myVal == [] or myVal == None:                        # check if value is empty
                            propval += 1;
                            errorcheck += 1;
                            fileoutput.write('ERROR: Component[{0}]: 
                             Primitive Property[{1}]: has missing value\n'.\
                                format(root.get('Name'), prop.get('Name')))
                        elif basicAtt1=='Real':
                            if basicAtt3 == [] or basicAtt3 == None:
                                warningcheck += 1;
                                fileoutput.write('WARNING: Component[{0}]: 
                                 Primitive Property[{1}]: No Units\n'.\
                                    format(root.get('Name'), prop.get('Name')))
                            if basicAtt2 ==  
                             '1':                                # check if 
                                             scalars are numbers
                                if not is_number(myVal):
                                    propval += 1;
                                    errorcheck += 1;
                                    fileoutput.write('ERROR: 
                                     Component[{0}]: Primitive 
                                      Property[{1}]: has incorrect data 
                                       type\n'.\
                                        format(root.get('Name'), prop.get('Name')))
                            else:
                                if is_number(basicAtt2):
                                    myLength = int(basicAtt2)
                                else:
                                    myDim=basicAtt2.split('x')
                                    myLength = int(myDim[0]) * int(myDim[1])
                                if not 
                                 is_list(myVal):                          # 
                                  check if matrix/vector is all numbers
                                    propval += 1;
                                    errorcheck += 1;
                                    fileoutput.write('ERROR: 
                                     Component[{0}]: Primitive 
                                      Property[{1}]: has incorrect data 
                                       type\n'.\
                                        format(root.get('Name'), prop.get('Name')))
                                if not myVal.count(',') + 1 == 
                                 myLength:        # check if matrix/vector 
                                  has right number of entries
                                    propval += 1;
                                    errorcheck += 1;
                                    fileoutput.write('ERROR: 
                                     Component[{0}]: Primitive 
                                      Property[{1}]: has dimension 
                                       error\n'.\
                                        format(root.get('Name'), 
                                         prop.get('Name')))
                        elif is_number(myVal) and basicAtt1=='String':
                            propval += 1;
                            warningcheck += 1;
                            fileoutput.write('WARNING: Component[{0}]: 
                             Primitive Property[{1}]: has real data and 
                              listed as string'.\
                                format(root.get('Name'), prop.get('Name')))
                                         
        propnames2.append(prop.get('Name'))
    return [warningcheck,errorcheck,propval,propnames2, concheck] ##propval




def check_properties(xmldoc, statsData, fileoutput):  # checks that the 
     resource link in the XML file are valid
    """Return true or false based on whether the resource objects in the 
       XMl point to valid files"""
    root = xmldoc.getroot()
    

    props = root.findall("Property")
    propnames = [p.get('Name') for p in props]
    if len(propnames) != len(set(propnames)):
        warningcheck += 1;
        print 'ERROR: Component[{0}]: duplicate property 
         names'.format(root.get('Name'))

    propval = 0;
    warningcheck = 0;
    errorcheck = 0;
    concheck = 0;
    
    propnames2 = list()

    for child in props:
        notes = check_property(child, child.get(XSI + 'type'), root, 
         propval, propnames2, errorcheck, warningcheck, fileoutput, 
          concheck)
        propval = notes[2];
        errorcheck = notes[1];
        warningcheck = notes[0];
        propnames2 = notes[3];
        concheck = notes[4];


    statsData.append(len(props))                                    ## 
     number of property errors in file
    print 'number of property errors =', propval
    statsData.append(propval)
    ##print True

    propval2 = len(propnames2) - len(set(propnames2))
    statsData.append(propval2)
    statsData.append(errorcheck)
    statsData.append(warningcheck)
    statsData.append(concheck)

    return True


def check_connectors(xmldoc, statsData):  # checks that the resource link 
 in the XML file are valid
    """Return true or false based on whether the resource objects in the 
       XMl point to valid files"""
    root = xmldoc.getroot()
    conns = root.findall("Connector")
    count = 0

    if (conns is None) or (len(conns) == 0):
        print 'WARNING: Component[{0}: No 
         connectors'.format(root.get('Name'))
    else:
        names = [c.get('Name') for c in conns]
        if len(names) != len(set(names)):
            print 'ERROR: Component[{0}]: duplicate connector names'.\
                format(root.get('Name'))

        for child in conns:
            roles = child.findall('Role')
            if (roles is None) or (len(roles) == 0):
                print 'WARNING: Component[{0}]: Connector[{1}] has no 
                 content'.format(root.get('Name'), child.get('Name'))
            for r in roles:
                source = r.attrib.get('PortMap')
                if 'cad' in source:
                    count += 1
            
    statsData.append(len(conns))
    statsData.append(count)
    return True


def parse_args():    # parses the command line arguments
    """Returns a list of program arguments"""
    parser = OptionParser()
    parser.add_option("-p", "--path", dest="path",
                  help="path to directory containing component XML's", 
                   metavar="PATH")
    parser.add_option("-x", "--schema", dest="xsd",
                  help="path to the component XML schema", metavar="SCHEMA")
    parser.add_option("-q", "--quiet",
                  action="store_false", dest="verbose", default=True,
                  help="don't print status messages to stdout")

    (options, args) = parser.parse_args()
    if options.path is None:
        parser.error("missing required path argument")
        return None
    if options.xsd is None:
        parser.error("missing required schema argument")
        return None

    print 'XML Path: {0}'.format(options.path)
    print 'Schema File: {0}'.format(options.xsd)
    
    return options


    

def main(argv=sys.argv):
    options = parse_args()
    compxmls = list_cxmls(options.path)
    stats = list()
    filecount=0;
    fileoutput = open('Python_Log.txt','w')

    for x in compxmls:
        # parse the XML and validate
        statsData = list()
        xmldoc = schema_validate(x, options.xsd)
        if xmldoc is None:
            print 'ERROR: Component XML[{0}]: failed schema 
             validation'.format(x)
            statsData.extend([x, '<a href="' + x + '">XML File</a>', False, 
             None, False, False, False, False, 0, 0])  # Schema invalid
        else:
            xmldoc = etree.parse(x)
            statsData.extend([xmldoc.getroot().get('Name'), '<a href="' + x 
             + '">XML File</a>', True])  # Schema valid
            check_classification(xmldoc, statsData)
            check_resources(xmldoc, options.path, statsData)
            check_domain_models(xmldoc, statsData)
            check_properties(xmldoc, statsData, fileoutput)
            ##check_parameters(xmldoc, statsData)
            check_connectors(xmldoc, statsData)
            #print len(statsData)
        stats.append(statsData)

    
    fileoutput.close()

    headers = ['Component', 'XML', 'Schema Valid', 'Classification', 
     'Resources Valid', 'Has CAD',
               'Has Modelica', 'Has Manufacturing', 'Num Properties', 
                'Empty Properties','Duplicate Names' ,'Property 
                 Errors','Property Warnings','CAD Derived Properties','Num 
                  Connectors','CAD References in Connectors']
    tld = tablib.Dataset(*stats, headers=headers)
    tld2 = tablib.Dataset(headers=["Category", "Instances"])
    for k,v in catCounts.iteritems():
        tld2.append([k,v])

    stats_file = open("stats.html", "w")
    stats_file.write("<html>")
    stats_file.write('<!-- DataTables CSS --> \
<link rel="stylesheet" type="text/css" 
 href="http://ajax.aspnetcdn.com/ajax/jquery.dataTables/1.9.4/css/jquery.dat
  aTables.css"> \
 \
<!-- jQuery --> \
<script type="text/javascript" charset="utf8" 
 src="http://ajax.aspnetcdn.com/ajax/jQuery/jquery-1.8.2.min.js"></script> \
 \
<!-- DataTables --> \
<script type="text/javascript" charset="utf8" 
 src="http://ajax.aspnetcdn.com/ajax/jquery.dataTables/1.9.4/jquery.dataTabl
  es.min.js"></script> \
\
<script type="text/javascript"> \
$(document).ready(function() { \
    $("table").dataTable(); \
} ); \
</script> \
')

    stats_file.write("<body>")
    stats_file.write('<p style="font-weight: bold;">Component Instance 
     Report</p>')
    stats_file.write('<div id="component_table">')
    stats_file.write(tld.html)
    stats_file.write('</div>')
    stats_file.write('<div class="clear"/><br><br>')
    stats_file.write('<p style="font-weight: bold;">Category Count 
     Report</p>')
    stats_file.write('<div id="category_table">')
    stats_file.write(tld2.html)
    stats_file.write('</div>')
    stats_file.write('<div class="clear"/>')
    stats_file.write("</body>")
    stats_file.write("</html>")
    stats_file.close()

    return 0

if __name__ == "__main__":
    sys.exit(main())    
\end{MyVerbatim}

\section{Component Class-based Unit Tests}
An automated unit testing framework performs in-class conformance check on components. As depicted in figure \ref{Unit_test_framework} the unit-test framework automatically instantiates a component in a testbench model constructed apriori in META tools. The framework executes CyPhy model composer tools, and executes the generated artifacts (Modelica simulation,  CAD Constraint engine, etc.). The results of the tests are post processed to determine if the component satisfies class-specific conformance requirements specified through the unit tests. 

\begin{figure}
\includegraphics*[width=\textwidth]{Unit_test_framework}
\caption{Unit Testing Framework}
\label{Unit_test_framework}
\end{figure}

% add a unit test example and explain what its testing

\chapter{Reference Model}
\section{Component Descriptor}
\section{Package Contents}

