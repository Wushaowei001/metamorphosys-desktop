\section{General Modeling Conventions}
\subsection{Domain Models at Multiple Levels of Fidelity}
An AVM Component Model may come with multiple models within a given domain (CAD, Modelica, etc), each capturing component behavior or characteristics at differering levels of fidelity.

For details on how these "alternatives" are captured in the AVM Component Model, refer to the \textit{Domain Model Alternatives} section of \textit{\autoref{Domain_Model_Alternatives}: \nameref{Domain_Model_Alternatives}}.

\subsection{CAD}
AVM Components may represent objects at varying levels of complexity. An AVM Component may represent a single part, such as a metal plate. It could represent a collection of objects that we could think of as an assembly in the real world, such as a diesel engine, but be captured by a single CAD part file. It could also represent the same assembly, but captured by a CAD assembly model with subordinate part files.

Depending on how the Component is modeled, certain conventions are required to ensure that all of the artifacts associated with a Component Model carry consistent identities. This is especially the case when analysis tools need to cross-reference elements in different domains, such as establishing references between CAD geometry models and detailed manufacturing models.

\subsection{Component Represented by a Single Part Model}
CAD representations of the component must capture it's geometry as a single part file. If multiple CAD models are provided, they are assumed to be of alternate levels of detail, or specialized for different purposes.

\subsection{Component Represented by an Assembly of Parts}
For some components, like brackets, the geometric representation may be provided as a CAD Assembly with a number of subordinate parts. Naming conventions are used to establish the unique identity of each instance of each part.

\subsubsection{CAD}
Each instance of each part in the assembly must have a unique part file. Even if two part instances are identical, they must be represented as unique part files with unique names.

Using an example of a bracket, featuring two identical plates welded together, we expect to see 3 CAD files: one for the assembly, and one for each plate.

\subsubsection{Manufacturing}
Each instance of each part must have a unique Manufacturing model. Each should be named to match the corresponding CAD part file. Any references to individual parts within these Manufacturing models must use this name also.

Using the same example of a bracket, featuring two identical plates welded together, we expect to see at least 3 Manufacturing models: one for each plate (identified by their respective unique names), and one representing the weld between them (referring to part1 and part2 using those unique part names).

\subsubsection{Important Note}
Since identities are established by file naming conventions, this means that only ONE of each component of this type may be included in a design. Using the bracket example, only one instance of that bracket may be instantiated. Otherwise, there will be name and identity conflicts.

This is a known limitation of this approach (VU META and iFAB are aware of this). The future plan is to find a reliable way to modify the part identities to include a notion of the component instance. This will probably take the form of an intelligent script that knows how to find the key identities and consistently prepend a "component instance" string to them.


