\Section{Introduction}

A number of platform technologies aim to support high-confidence distributed
control systems design.

1. Give a short general introduction, so we know where the problem and solution
   fit in the research field.

CPS -> High-confidence -> distributed control systems -> real-time scheduling

Designing software for complex safety-critical systems poses unique challenges. 
Control theory, distributed software design principles, and real-time scheduling
techniques intersect in this design domain.  Each of these design disciplines 
imposes a different set of correctness constraints.  Researchers strive to make
the techniques from these areas work together in a meaningful way.

2. Introduce the specific problem and explain what is difficult about it.

More specifically 
ARINC + distributed objects with multiple communication semantics

Distributed object computing relies on well-defined component interfaces and 
on computational abstractions for networked data communications in order to 
simplify the specification and deployment of large networked software systems.
Distributed systems are specified as a design-time component model and a 
runtime framework for deploying designs.  These techniques allow distributed 
software designs to scale to larger problems.   In avionics designs the 
ARINC-653 standard defines runtime services to support safety-critical designs,
but does not deal with distribution concerns (is that true?).  The APEX kernel 
associated with the standard runs on top of a real-time operatings system (RTOS)
and defines how processes are grouped into spatially and temporally distributed 
partitions. Numerous control design analysis techniques support these 
high-confidence design problems, but lie beyond the scope of the current 
discussion.  

High-confidence design and analysis techniques are necessarily formal, and are 
specified to a fine level of detail.  Perhaps the biggest challenge in using
multiple techniques together lies in reconciling those details, and filling in 
the gaps where their intersection is not complete.  Consider the case of a
distributed object system running real-time communicating tasks in ARINC-653
partitions, as in the ARINC-653 Component Model \cite{}.  The first challenge
would be the creation of a runtime environment that supports both distributed
object-style specification and deployment as well as timely partitioned 
execution. The second challenge would be the joint analysis of both frameworks 
with respect to schedulability.  

Avionics -> Partitioned execution + real-time tasks within partitions
Avionics -> Distributed messaging -> non-blocking (time-triggered) + blocking messages


3. Briefly describe our proposed solution to the problem, and our specific 
   contributions to the proposed solution.

Easwaran showed how to do Partitions + EDF on a single processor.

We need to be able to do fixed priority scheduling within partitions, 
time-triggered message scheduling between partitions on different processors,
and guarantee that blocking messages between tasks in any partitions will not
cause deadline violations.

We propose to extend the hierarchical scheduling framework originally proposed
by Shin and Lee, and extended by Easwaran and Lee.

1. Easwaran's model treats statically scheduled partitions as a periodic 
   supply to the tasks within the partitions.
2. Partition offset bounds must respect message dependency requirements.
3. We assume that blocking messages are sent with worst-case latency -
   the message is sent at the start of the sending task (which blocks), the
   receiver gets it at the start of its execution time, and runs to completion
   before sending a response to unblock the sender. 

We are extending our constraint-based scheduling tool to handle the case 
described above.

Cover related work here in brief.

Old Notes:

\SubSection{Problems}

Semantic gap -- mixed design paradigms, and multiple communication models.

- Task execution models: Giotto: ARINC-653
-Communication Message Scheduling: TTP/C TTEthernet - to some extent AFDX
- Issue: combining them together and solving an integrated problem. 

Examples: 

\begin{enumerate}
\item Distributed tasks and messaging with time- and event-triggered execution
models.  Consider the time-triggered ethernet communication protocol.
\item ARINC-653 partitioned execution does not fully specify the semantics of
intermodule communication.  We consider the case where real-time tasks within a 
partition make blocking calls to other tasks in the system.  Assuming that the
dependencies are known \emph{a priori}, the task and partition offsets must be 
calculated to satisfy the communication constraints.  This example assumes an
AFDX-style communication model.
\end{enumerate}

\SubSection{Contributions}

Unified scheduling tool, including:

\begin{enumerate}

\item Integrated task and message scheduling:

\begin{itemize}
\item Blocking dependencies

Partitioned execution and communication are critical for secure and 
fault-tolerant real time systems design.  However, distribution of computation
over a network complicates partition scheduling, especially in the case of
blocking calls (e.g. for remote method invocations).   One solution to this
problem is the use of synchronized partitions and time-triggered data 
communications.  Previous execution models rely on non-blocking 
communications to enhance fault tolerance, and logical execution time 
constraints to simplify the calculation of the static message schedule.

We propose to increase the flexibility of this design paradigm by adding 
blocking calls to partitions.  We assume that all partition schedules in the
distributed system are synchronized, to ensure proper overlap of blocking
messages between partitions.  Obviously, the burden of designing the system
specifications to eliminate conflicts will fall to the designers.  We will 
also address static \emph{a priori} deadlock detection from the specification.

\item Offset determination
\end{itemize}

\item Two-level scheduling model

\begin{itemize}
\item Partitioned execution
\item Mixed time-triggered and event-triggered approach
\end{itemize}

\item Flexible communication bus model

\item Extensions of the incremental approach

\end{enumerate}


