\section{Conclusions}
\label{S:conclusions}
From Fig.~\ref{F:four_tank_nonmin_phase_x2} it is clear that IDA-PBC
can achieve superior tracking performance when compared to the
decentralized controller for the \nonminimum-phase four tank process.
Unlike the decentralized controller we 
evaluated, the IDA-PBC provides both explicit constraints on allowable
controller gains and set-point trajectories $x^*$ which can be
enforced at run-time.  We clarified how to correctly determine if
$Q_d < 0$ and verified correct constraints on $k_i$ and
$k_{\mathsf{I}}$ for the four tank process which were
incorrectly determined in
\cite{johnsen07:_inter_and_dampin_assig_passiv} which allowed us to
provide new results showing a working controller for the non-minimum phase
four-tank system.  We further improved
system resilience by implementing a 
feasible integrator anti-windup compensator as demonstrated in the
full-scale step responses of the two-tank process depicted in 
Fig.~\ref{F:two_tank_x1} and Fig.~\ref{F:two_tank_x2}.  The explicit 
solution for $u^*=-l$ and the uncontrollable components of $x^*$
clearly provide visual indications about actuator degradation and
uncertainty in $\gamma$ for the coupled tank processes we studied.
Finally, we demonstrated that the bilinear transform can be used to
achieve moderately large sample times $T_s = .1$ second which will
reduce both computational and communication demands.