\section{Simulation Results}
\label{S:simulations}
In evaluating our proposed solution we shall take a closer look at
system performance for the modified two tank process described in
\cite{johnsen07:_inter_and_dampin_assig_passiv}.   Specifically we
will compare our controller with the additional integrator and
corresponding integrator anti-windup compensator to the original
proportional feed-back controller presented in
\cite{johnsen07:_inter_and_dampin_assig_passiv}.  We will see that the
integrator is able to effectively compensate for both actuator
degradation and flow-ratio uncertainty.  In addition the integrator
anti-windup compensator works sufficiently well in prohibiting
significant oscillatory behavior when operating at the systems
limits.  Next we will compare IDA-PBC performance to the decentralized
controllers presented in \cite{johansson00:_quadr_tank_proces} for
both the minimum and non-minimum phase cases.  We will see that the
IDA-PBC is both comparable for the minimum phase case while being
vastly superior for the non-minimum phase case.
\subsection{Two Tank Process}
For the two tank process the system operating parameters are as
follows: $A_1=50.3$ cm$^2$, $A_2=28.3$ cm$^2$, $a_1=.233$ cm$^2$, $a_2=.127$
cm$^2$, $\gamma=.4$, $\delta_{\gamma}=.75$, $k_u=.75$, $u_{\min}=0$,
$u_{\max}=100$, $x_1(0)=15$ cm, $x_2(0)=\left (
  \frac{(1-\delta_{\gamma}\gamma)a_1}{a_2} \right )^2x_1(0)$, $T_s=1$
second and $g=981$ cm/s$^2$.  Set-point trajectories for both the two
tank and four tank processes are smoothed using a discrete-time filter
which results from applying the bilinear transform to the
corresponding continuous time-time filter model $H_{\traj}(s)=\frac{\omega_{\traj}^2}{s^2 +
  2\zeta_{\traj}\omega_{\traj} s + \omega_{\traj}^2}$.  We compared
our controller for the two tank process to the controller presented in  
\cite{johnsen07:_inter_and_dampin_assig_passiv} which lacks
the additional integrator term $x_{\mathsf{I}1}$ to compensate for
model uncertainty.  Specifically $u=-[k_1,(1-\gamma)k_2]\hat{x} +
a_1\sqrt{2gx_1^*}$ in which $k_1 = 10$ and  $k_2 = 1.01 \frac{A_2
  k_1}{A_1 4} = 1.4206$.
\begin{figure}
  \centering
  \includegraphics*[width=3.5in,height=2in, viewport=20 10 400 300]{figures/two_tank_x1}
  \caption{Two tank process $x_1(t)$.}
  \label{F:two_tank_x1}
\end{figure}
\begin{figure}
  \centering
  \includegraphics*[width=3.5in,height=2in, viewport=15 10 415 300]{figures/two_tank_x2_u}
  \caption{Two tank process $x_2(t)$ and $u(t)$.}
  \label{F:two_tank_x2}
\end{figure}
%\begin{figure}
%  \centering
%  \includegraphics*[width=3.5in,height=2in, viewport=20 10 400 300]{figures/two_tank_u}
%  \caption{Pump flow command $u(t)$ for two tank process.}
%  \label{F:two_tank_u}
%\end{figure}
\subsection{Four Tank Process}
The reference for comparison is the decentralized controller (DC) used
to control the four tank process \cite{johansson00:_quadr_tank_proces}.  Specifically two
PI-controllers were used such that $U_l(s) = K_l \left ( 1 + \frac{1}{T_{il}}
  s \right )(X_l^*(s)-X_l(s)),\ l\ \in\ \{1,2\}$ in which
$(K_1=3.0,T_{i1}=30)$ and $(K_2=2.7,T_{i2}=40)$ for the minimum-phase
case and $(K_1=1.5,T_{i1}=110)$
and $(K_2=-.12,T_{i2}=220)$ for the \nonminimum-phase case.  The remaining parameters for the
process are as follows: i) $A_1 = A_3 = 28$ cm$^2$,  $A_2 = A_4 = 32$
cm$^2$, $a_1 = a_3 = 0.071$ cm$^2$, $a_2 = a_4 = 0.057$ cm$^2$,
$u_{1\min} = u_{2\min} = 0$, $u_{1\max} = u_{2\max} = 100$, $k_u =
0.8$; ii) either $\gamma_1 = 0.7$, $\gamma_2 = 0.6$, $x_1(0) = 12.4$ cm,
$x_2(0) = 12.7$ cm, $x_3(0) = 1.8$ cm, and $x_4(0) = 1.4$ cm for the
minimum-phase case or $\gamma_1 = 0.43$, $\gamma_2 = 0.34$, $x_1(0) =
12.6$ cm, $x_2(0) = 13.0$ cm, $x_3(0) = 4.8$ cm, and $x_4(0) = 4.9$ cm
for the \nonminimum\ phase case.

For the minimum-phase example (Figs. \ref{F:four_tank_min_phase_x1} and 
\ref{F:four_tank_min_phase_x2}) the IDA-PBC parameters are 
$k_3 = 100$, $k_4 = 100$, $\epsilon_1 = 0.75$, $\epsilon_2 = 0.75$, 
$\epsilon_{\mathsf{I}1} = 0.4$, $\epsilon_{\mathsf{I}2} = 0.4$,
$T_s=0.1$ s.  For the \nonminimum-phase example
(Figs. \ref{F:four_tank_nonmin_phase_x1} and
\ref{F:four_tank_nonmin_phase_x2}) the IDA-PBC parameters are  
$k_3 = 5$, $k_4 = 5$, $\epsilon_1 = 0.75$, $\epsilon_2 = 0.75$, 
$\epsilon_{\mathsf{I}1} = 0.75$, $\epsilon_{\mathsf{I}2} = 0.75$.  In
which the remaining controller coefficients are computed using the
following relationships $k_{\mathsf{I}1} = \epsilon_{\mathsf{I}1}
\frac{4}{A_1}$, $k_{\mathsf{I}2} = \epsilon_{\mathsf{I}2}
\frac{4}{A_2}$, $k_1 = \epsilon_1 \frac{(4-k_{\mathsf{I}1} A_1)
  k_3}{A_3}$ and $k_2 = \epsilon_2 \frac{(4-k_{\mathsf{I}2} A_2)
  k_4}{A_4}$.
% Table for the two cases
%\begin{table*}[htb]
%  \centering
%     \begin{tabular}[width=2.5in]{@{\extracolsep{\fill}} | c | c | c | }
%        \hline
%        \multicolumn{3}{|c|}{Four Tank Controller Parameters} \\
%        \hline
%         & Min Phase & Non-min Phase \\
%        \hline \hline
%        $\gamma_1$ & 0.7 & 0.43 \\
%        \hline
%        $\gamma_2$ & 0.6 & 0.34 \\
%        \hline
%        $k_3$ & 100 & 5 \\
%        \hline
%        $k_4$ & 100 & 5 \\ 
%        \hline
%        $\epsilon_1$ & 0.75 & 0.75 \\
%        \hline
%        $\epsilon_2$ & 0.75 & 0.75 \\
%        \hline
%        $\epsilon_{\mathsf{I}1}$ & 0.4 & 0.75 \\
%        \hline
%        $\epsilon_{\mathsf{I}2}$ & 0.4 & 0.75 \\
%        \hline
%        $T_s$ & 0.1 s & 1 s \\
%        \hline
%        $x_1(0)$ & 12.6 & 12.6  \\
%        \hline
%        $x_2(0)$ & 12.7 &  13.0 \\
%        \hline
%        $x_3(0)$ & 1.8 & 1.8  \\
%        \hline
%        $x_4(0)$ & 1.4 &  1.4 \\
%        \hline
%     \end{tabular}
%\end{table*}
\begin{figure}
  \centering
  \includegraphics*[width=3.5in,height=2in, viewport=20 10 410 300]{figures/min_phase_x1_u1}
  \caption{Minimum-phase four tank process $x_1(t),\ u_1(t)$.}
  \label{F:four_tank_min_phase_x1}
\end{figure}
\begin{figure}
  \centering
  \includegraphics*[width=3.5in,height=2in, viewport=20 10 410 300]{figures/min_phase_x2_u2}
  \caption{Minimum-phase four tank process $x_2(t),\ u_2(t)$.}
 \label{F:four_tank_min_phase_x2}
\end{figure}
\begin{figure}
  \centering
  \includegraphics*[width=3.5in,height=2in, viewport=20 10 410 300]{figures/nonmin_phase_x1_u1}
  \caption{\Nonminimum-phase four tank process $x_1(t),\ u_1(t)$.}
  \label{F:four_tank_nonmin_phase_x1}
\end{figure}
\begin{figure}
  \centering
  \includegraphics*[width=3.5in,height=2in, viewport=20 10 410 300]{figures/nonmin_phase_x2_u2}
  \caption{\Nonminimum-phase four tank process $x_2(t),\ u_2(t)$.}
  \label{F:four_tank_nonmin_phase_x2}
\end{figure}
