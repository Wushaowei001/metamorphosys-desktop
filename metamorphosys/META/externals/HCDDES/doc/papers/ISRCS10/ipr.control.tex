\section{IDA-PBC, Integrator Anti-Windup \& Discrete Time Control}
\label{S:control}%Nicholas
Our primary concern is to determine a desired state trajectory $x^*$
which may be a function of a smaller subset of desired {\em
  independently-controllable} states $x_{\mathsf{c}}^*$ and a corresponding
control law $u=\beta(x,x_{\mathsf{I}},x^*)\ \in\ \mathbb{R}^m$ ($x,x^* \in\
\mathbb{R}^n$ and $x_{\mathsf{c}}^*, x_{\mathsf{I}} \in
\mathbb{R}^{p}$ in which $p \leq n$ is the number of
independently controllable states) for the following input-affine system 
\begin{equation}
\label{E:ias}
\dot{x} = f(x) + g(x)u 
\end{equation}
augmented with $p$ additional (non-linear) integrator
states $x_{\mathsf{I}}$ to account for system uncertainty such that 
\begin{align*}
\dot{x}_{\mathsf{I}} &= k_{\mathsf{I}}\left ( f_{\mathsf{I}} (x_{\mathsf{c}}) -
  f_{\mathsf{I}} (x_{\mathsf{c}}^*) \right )\\
&= \diag \{
k_{\mathsf{I}1}\left ( f_{\mathsf{I1}} (x_{\mathsf{c}1}) -
  f_{\mathsf{I1}} (x_{\mathsf{c}1}^*) \right ),\dots,k_{\mathsf{I}p}\left ( f_{\mathsf{I}p} (x_{\mathsf{c}p}) -
  f_{\mathsf{I}p} (x_{\mathsf{c}p}^*) \right )\}.
\end{align*}
N.B. it may not be necessary to introduce $p$ integrators in order to
account for system uncertainty, however, in order to simplify
discussion we will assume that the system studied is asymptotically
stable for each controllable state.  This will typically require an
additional integral term in order to account for system uncertainty
and actuator degradation.  In addition, we desire to implement an
anti-windup control law which will improve system resilience when
either the control actuator deteriorates or other system parameters
deviate substantially such that actuator saturation occurs.  Finally,
we consider a discrete-time implementation of our control-law in which
we use a bilinear-transform to approximate the integral and
anti-windup control law.  Section~\ref{S:ida_pbc} provides an overview
of IDA-PBC which will allow us to effectively control (non)-linear
systems and achieve large operating ranges while being able to
quantify system degradation. Section~\ref{S:anti_windup} presents our
integrator anti-windup compensator which typically improves system
resilience.  Finally, Section~\ref{S:DT_imp} presents our
discrete-time implementation which we found to work exceptionally well
by allowing large sampling times $T_s$.
\subsection{IDA-PBC}
\label{S:ida_pbc}
IDA-PBC is concerned with rendering our input affine system \eqn{E:ias} 
with augmented integrator control law to have the following final form
in terms of the gradient of the desired Hamiltonian $H_d(x,x_{\mathsf{I}},x^*)$:
\begin{equation}
\label{E:match_eq}
\begin{bmatrix}\dot{x}\\\dot{x}_{\mathsf{I}}\end{bmatrix} = Q_d(x,x_{\mathsf{I}}) \grad H_d(x,x_{\mathsf{I}},x^*) 
\end{equation}
in which $Q_d(x,x_{\mathsf{I}}) \in\ \mathbb{R}^{(n+p) \times (n+p)}$ is negative definite
($Q_d(x,x_{\mathsf{I}}) < 0$) and the operation $\grad = [\frac{\partial}{\partial
  x_1},\dots,\frac{\partial}{\partial x_n},\frac{\partial}{\partial x_{\mathsf{I}1}},\dots,\frac{\partial}{\partial x_{\mathsf{I}p}}
  ]\tr$.  In order to guarantee that $x=x^*$ and $x_{\mathsf{I}}=0$ at steady-state (for
the ideal model-matching case) the control law $\beta(x,x_{\mathsf{I}},x^*)$ should
guarantee that: 
\begin{gather*}
\grad H_d(x=x^*,x_{\mathsf{I}}=0,x^*) = 0 \text{ necessary}\\
\hessian H_d(x=x^*,x_{\mathsf{I}}=0,x^*) > 0 \text{ a sufficient condition.}
\end{gather*}
Although not required for controller synthesis, many physical systems,
such as robotic systems $Q_d = J_d - R_d$ in which
$J_d=-J_d\tr$ is a skew-symmetric matrix representing the
underlying network structure of the system whereas $R_d =
R_d\tr \geq 0$ describes the damping in the system
\cite{ortega02:_inter_and_dampin_assig_passiv}.  Such observations may
prove useful in choosing an initial Hamiltonian to begin control
design, however, by choosing $Q_d$ to simply be a constant negative
definite matrix an engineer can systematically determine a controller
$\beta(x,x_{\mathsf{I}},x^*)$ as originally described in
\cite{johnsen07:_inter_and_dampin_assig_passiv}  and summarized here
(in which we add some additional discussion on determining $x^*$ from
$x_{\mathsf{c}}^*$ while improving upon the discussion in introducing
additional integrator terms).
\begin{enumerate}
\item Recall that the introduction of $p$ additional (non-linear)
  integrators results in an augmented state-space description in
  order to account for system uncertainty:
\begin{equation*}
\begin{bmatrix}\dot{x}\\\dot{x_{\mathsf{I}}}\end{bmatrix} = \begin{bmatrix}f(x) \\
 k_{\mathsf{I}}\left ( f_{\mathsf{I}} (x_{\mathsf{c}}) -
  f_{\mathsf{I}} (x_{\mathsf{c}}^*) \right )\end{bmatrix}+ \begin{bmatrix} g(x)\\
0
\end{bmatrix} u.
\end{equation*}
\item Select a candidate Hamiltonian $H(x,x_{\mathsf{I}})$ which
  depends on additional scaling terms $k_i$, $i\ \in \{1,\dots,n\}$
  such that 
\begin{equation*}
\begin{bmatrix}\dot{x}\\\dot{x}_{\mathsf{I}} \end{bmatrix}= Q_d \grad 
  H(x,x_{\mathsf{I}}) = \begin{bmatrix}f(x) \\
 k_{\mathsf{I}}\left ( f_{\mathsf{I}} (x_{\mathsf{c}}) -
  f_{\mathsf{I}} (x_{\mathsf{c}}^*) \right )\end{bmatrix}
\end{equation*}
 in which $Q_d\ \in\ \mathbb{R}^{(n+p) \times (n+p)}$ is a constant matrix.
\item Determine the conditions on $k_i$ and $k_{\mathsf{I}}$ such
  that $Q_d$ is negative definite.  We do this by verifying that the negative of
  the Hermitian of $Q_d$ ($-\hermitian\{Q_d\}=-\frac{1}{2}(Q_d\tr+Q_d)$)
  is positive definite using Sylvester's Criterion (the
  determinants of the leading principal submatrices of $-\hermitian\{Q_d\}$ are
  positive).
\item Determine a matrix $P \in \mathbb{R}^{(n+p) \times
    m}$ having columns spanning the null-space of 
  $g^{\bot}Q_d$ ($g^{\bot}Q_dP=0$ and normalized such that
  $g^{\dagger}Q_dP=-I$) in which $g^{\bot}(x)\ \in\
  \mathbb{R}^{(n-m)\times (n+p)}$ is the maximum rank
  left annihilator of $g(x)$ such that $g^{\bot}(x)[g(x)\tr,0\tr]\tr=0$ and
  $g^{\dagger}(x) \ \in\
  \mathbb{R}^{m \times (n+p)}$ is the left-inverse of $g(x)$ such that
  $g^{\dagger}(x)[g(x)\tr,0\tr]\tr=I$ ($P$ is used to compute the characteristic $z
  = P\tr [x\tr,x_{\mathsf{I}}\tr]\tr$).
\item Denoting $\tilde{x}=[(x-x^*)\tr,(x_{\mathsf{I}}-0)\tr]\tr$,
  $\tilde{z}=P\tr \tilde{x}$ and choosing the desired Hamiltonian $H_d$ 
  to be of the following form:
\begin{equation*}
H_d(x,x_{\mathsf{I}},x^*)=H(x,x_{\mathsf{I}}) + \frac{1}{2} \tilde{z}\tr Q
\tilde{z} + l\tr z 
\end{equation*}
in which $Q\ \in\ \mathbb{R}^{m \times m}$ $Q=Q\tr > 0$.  The corresponding gradient is 
\begin{equation*}
\grad H_d (x,x_{\mathsf{I}},x^*)=\grad H(x,x_{\mathsf{I}}) + PQP\tr \tilde{x} +
P l.
\end{equation*}
When $x=x^*$ and $x_{\mathsf{I}}=0$ the gradient simplifies to
\begin{equation*}
\grad H_d (x=x^*,x_{\mathsf{I}}=0,x^*)=\grad H(x=x^*,x_{\mathsf{I}}=0)
+ P l
\end{equation*}
which we use to determine $l$ and the uncontrolable parts of $x^*$
which are not part of $x^*_c$ such that $\grad H_d
(x=x^*,x_{\mathsf{I}}=0,x^*)=0$.  In addition we verify that the
corresponding Hessian is positive definite for $x=x^*$  and
$x_{\mathsf{I}}=0$
\begin{equation*}
\hessian H_d (x=x^*,x_{\mathsf{I}}=0,x^*)=\hessian H(x=x^*,x_{\mathsf{I}}=0)+P Q P\tr.
\end{equation*}
\item Finally we determine our control law $\beta(x,x_{\mathsf{I}},x^*)$:
\begin{gather*}
\beta(x)=g^{\dagger} \left \{ Q_d \grad H_d (x,x_{\mathsf{I}},x^*) -  \begin{bmatrix} f(x)\\ k_{\mathsf{I}}\left ( f_{\mathsf{I}} (x_{\mathsf{c}}) -
  f_{\mathsf{I}} (x_{\mathsf{c}}^*) \right ) \end{bmatrix} \right \} \\
\beta(x,x_{\mathsf{I}},x^*)=g^{\dagger} \left \{ Q_d P \left ( QP\tr \tilde{x} +
 l \right )  \right \}
\end{gather*}
{\em Since $P$ was normalized such that $g^{\dagger}Q_dP=-I$} then our
  final control law has the following simplified form:
\begin{gather*}
\beta(x,x_{\mathsf{I}},x^*)=-K_p \tilde{x} + u^*\\
\text{in which } K_p = QP\tr,\ u^* = -l.
\end{gather*}
\end{enumerate}
\subsection{Integrator Anti-Windup Compensator}
\label{S:anti_windup}
In \cite{johnsen07:_inter_and_dampin_assig_passiv} the authors addressed integrator wind-up issues by simply setting
$\dot{x}_{\mathsf{I}}=0$ if $\beta(x,x_{\mathsf{I}},x^*) > u_{\max}$ or $\beta(x,x_{\mathsf{I}},x^*) <
u_{\min}$.  Unfortunately this ad hoc solution caused the simulation
to halt when evaluating their control law for the two tank and
four tank processes when actuator saturation occurred.  When the
overall system to be controlled is asymptotically 
stable a more effective approach is to introduce an additional
feed-back term which attempts to approximate the system dynamics as if
saturation has not occurred
\cite{fu10:_feedb_therm_contr_for_real_time_system}.  Since the
integrator dynamics can not describe the case when $x < 0$ we will use
a linear approximation of the system with respect to the desired
trajectory components $x^*$.  In addition the matrix $g(x)$ for the
systems studied do not depend on $x$.  We denote the Jacobian of
$f(x)$ as $A(x)$ such that 
\begin{equation*}
A(x)=\begin{bmatrix}
\frac{\partial f_1(x)}{\partial x_1} & \dots & \frac{\partial
f_1(x)}{\partial x_n}\\
\vdots & \vdots & \vdots\\
\frac{\partial f_n(x)}{\partial x_1} & \dots & \frac{\partial
f_n(x)}{\partial x_n}
\end{bmatrix}.
\end{equation*}
 Specifically we modify the integrator dynamics as follows:
\begin{equation*}
\dot{x}_{\mathsf{I}} = k_{\mathsf{I}} 
  \left ( f_{\mathsf{I}}(x_{\linc}) - f_{\mathsf{I}}(x^*_c) \right ),\
  x_{\linc} = \bar{x}_c + x_c 
\end{equation*}
in which $x_{\linc}$ is further constrained to gaurantee that
$f_{\mathsf{I}}(x_{\linc})$ is valid
(ie. if $f_{\mathsf{I}}(x_{\linc})=\sqrt{x_{\linc}}$ then if $x_{\linc}
< 0$ set $x_{\linc}=0$).  $\bar{x}_c$ are the appropriate components of $\bar{x}$ which
are determined from the following dynamic relationship
\begin{gather*}
\dot{\bar{x}} = A(x^*) \bar{x} + g \bar{u}\\
\bar{u} = u - \sat(u,u_{\min},u_{\max}),\ u =
\beta(x,x_{\mathsf{I}},x^*)\\
\sat(u,u_{\min},u_{\max})=\begin{cases}
u\ \text{if } u_{\min}\ \leq u\  \leq \ u_{\max}, \\
u_{\min}\ \text{if } u\ < u_{\min}, \\
u_{\max}\ \text{otherwise.}
\end{cases}
\end{gather*}

We observe that for the linear time-invariant (\LTI) case that if 
exact knowledge of the plant dynamics are given such that both
$A(x) x=f(x)$ and the exact saturation model is
known then the closed-loop dynamics considered for stability are
identical to those considered when actuator saturation is not
considered.  Therefore stability is unaffected using our proposed
control scheme for the \LTI\ case.  Analysis for the non-linear case
is clearly much more involved and worthy of future study.  However,
for the non-linear tank processes studied, using the jacobian $A(x)$
to approximate the plant dyanmics was sufficient to prevent integrator
wind-up while maintaining stability.
\subsection{Discrete-Time Implementation}
\label{S:DT_imp}%
Observing the final control-law for $\beta(x,x_{\mathsf{I}},x^*)$ and
noting the form for the actuator anti-windup structure in computing
$\dot{\bar{x}}$ and $\dot{x}_{\mathsf{I}}$.  Most of the control
components consists of a standard matrix multiplication and the
corresponding (non-linear) integral terms can be approximated by
applying either a matched pole-zero or bilinear transform
\cite{franklin06:_feedb_contr_of_dynam_system}.  The bilinear 
transform is preferred because it preserves the passivity properties of the
integration term which typically allows for longer sample and hold
times than if a matched pole-zero method was used.  Thus reducing
communication bandwidth and typically reducing sensitivity to time
delay jitter.  We note that the integer $k$ shall be related to time
$t$ and sampling rate $T_s > 0$ as follows $k=\lfloor \frac{t}{T_s}
\rfloor$ for all $t \geq 0$.  Specifically, our discrete-time
implementation is as  follows:
\begin{gather*}
u(k) = \beta(x(kT_s),x_{\mathsf{I}}(k-1),x^*(k))\\
\bar{u}(k) = u(k) - \sat(u(k),u_{\min},u_{\max})\\
\dot{\bar{x}}(k)=A(x^*(k))\bar{x}(k)+ g \bar{u}(k)\\
\bar{x}(k) = \bar{x}(k-1) + \frac{T_s}{2} \left [
  \dot{\bar{x}}(k) + \dot{\bar{x}}(k-1) \right ]\\
x_{\linc}(k)=\bar{x}_c(k) + x_c(kT_s)\ \in\ [x_{\min},x_{\max}]\\
\dot{x}_{\mathsf{I}}(k)=k_{\mathsf{I}} \left [
  f_{\mathsf{I}}(x_{\linc}(k))-f_{\mathsf{I}}(x^*_c(k)) \right ]\\
x_{\mathsf{I}}(k) = x_{\mathsf{I}}(k-1) + \frac{T_s}{2} \left [
  \dot{x}_{\mathsf{I}}(k) + \dot{x}_{\mathsf{I}}(k-1) \right ]
\end{gather*}
in which $u(t)=u(k),\ t\ \in\ [kT_s,(k+1)T_s)$.  It is of future
interest to implement an observer for the plant-subsystem in order to
preserve dissipative properties in the discrete-time setting
\cite{stramigioli05:_sampl_hamil}.  Such an observer set-up could
potentially lead to a one-to-one mapping between the continuous-time
and discrete-time implementation in regards to satisfying
discrete-time stability for a given set of gain constraints.
