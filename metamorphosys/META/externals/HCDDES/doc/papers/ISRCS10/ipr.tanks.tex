\section{IDA-PBC of Coupled Tank Systems}
\label{S:tanks}%simple integrator example ???
A classic problem in the process control laboratory is to learn how to
control the height of columns of water for either a coupled two tank
\cite{astrom86:_teach_labor_for_proces_contr} or four tank process 
\cite{johansson00:_quadr_tank_proces}.  Each problem is particularly
interesting in that there are only half as many actuators (pumps) as
there are tanks of water whose heights there are to control.  Therefore
each process is under-actuated in which only half of the
tanks heights can be independently controlled and in particular for
the four tank process it is unclear how to systematically apply methods
such as back-stepping control to the process 
\cite{krstic95:_nonlin_and_adapt_contr_desig}.  However, as
\cite{johnsen07:_inter_and_dampin_assig_passiv} has demonstrated
IDA-PBC shows to be a promising tool to handle such a complicated
system.  In applying IDA-PBC to the four tank process we discovered
some {\em additional restrictions} are required on the independent
controllable states $x_c$ and the control gains $k_i$ and
$k_{\mathsf{I}}$.  Since we are particularly interested in
implementing a resilient integrator anti-windup compensator
Section~\ref{S:two_tank} recalls the coupled two tank process model which 
\cite{johnsen07:_inter_and_dampin_assig_passiv} modified to split the
flow into both the upper and lower tanks.  We will extend the
discussion on the control of this system by considering an additional
integral term to account for model uncertainty.
Section~\ref{S:four_tank} will recall the coupled four tank process model.
\subsection{Two Tank Process}
\label{S:two_tank}
The two tank process consists of a single gear-pump which provides
volumetric flow of a fluid from a bottom-reservoir proportional to the control
input $u$ whose flow is then split such that ideally $\gamma u$ is sent to a
lower-tank with cross-sectional area $A_1$ and drain-orifice area
$a_1$ which drains back into the bottom-reservoir.  The remaining
$(1-\gamma)u$ amount of fluid is sent to the upper-tank with
cross-sectional area $A_2$ and drain-orifice area $a_2$ which drains
back into the lower-tank.  The heights of the fluid in lower and
upper-tanks is denoted $x_1$ and $x_2$ respectively.  Using
Torricelli's Law the system dynamics have the following form
\begin{equation}
\label{E:two_tank}
\begin{bmatrix}
\dot{x}_1\\
\dot{x}_2
\end{bmatrix}=
\begin{bmatrix}
\frac{-a_1 \sqrt{2gx_1} + a_2 \sqrt{2gx_2}}{A_1} \\
-\frac{a_2 \sqrt{2gx_2}}{A_2} 
\end{bmatrix}+
\begin{bmatrix}
\frac{\delta_{\gamma} \gamma}{A_1}\\
\frac{1 - \delta_{\gamma} \gamma}{A_2}
\end{bmatrix} k_u u.
\end{equation}
For simplicity of discussion $k_u$ and $\delta_{\gamma}$ are nominally
considered to be equal to one, however we will perturb these values in
order to consider effects including pump degradation ($k_u < 1$) and
uncertainty in the flow-ratio $\gamma\ \in\ (0,1)$ such that $0 <
k_{\gamma} < \frac{1}{\gamma}$.  It should be clear from
\eqn{E:two_tank} that only the height of one of the tanks can be
independently controlled.  We will control the height of the
lower-tank $x_1=x_c$ and determine $x_2^*$ based on our desired height
$x_1^*$ and the system dynamics. In order to account for system
uncertainty we include the following integral control term:
\begin{equation}
\label{E:two_tank_I}
\dot{x}_{\mathsf{I}1} = a_1 \sqrt{2 g} \left ( \sqrt{x^*_1} -
  \sqrt{x_1} \right ).
\end{equation}
The corresponding Hamiltonian used to generate our controller is 
\begin{equation}
\label{E:two_tank_ham}
H(x) = \sum_{i=1}^2 \frac{2}{3} k_i a_i \sqrt{2g} x_i^{\frac{3}{2}} +
k_{\mathsf{I}1} a_1 \sqrt{x_1^*} x_{\mathsf{I}1}.
\end{equation}
Which results in:
\begin{equation*}
Q_d = \begin{bmatrix}
-\frac{1}{A_1 k_1} & \frac{1}{A_1 k_2} &  0\\
0 & -\frac{1}{A_2 k_2} &  0\\
\frac{k_{\mathsf{I}1}}{k_1} &  0 & -\frac{k_{\mathsf{I}1}}{k_1}
\end{bmatrix}
\end{equation*}
which is positive-definite iff $0 < k_1 < \frac{\left (4 - A_1 k_{\mathsf{I}1} \right
  )A_1k_2}{A_2}$, $0 < k_2 < \infty$ and $0 < k_{\mathsf{I}1} < \frac{4}{A_1}$;
and the remaining control related terms  
$P=[k_1,k_2(1-\gamma),k_1]\tr$, $l=-a_1\sqrt{2gx_1}$ and
$x_2^*=\frac{a_1^2(1-\gamma)^2}{a_2^2}x_1^*$.
\subsection{Four Tank Process}
\label{S:four_tank}
The four tank process consists of: i) lower-tanks Tank 1 and Tank 2
with respective fluid height $x_1$ and $x_2$ which we wish to control
$x_c=[x_1,x_2]\tr$; ii) upper-tanks Tank 3 and Tank 4 with fluid
height $x_3$ and $x_4$ such that $x=[x_c\tr,x_3,x_4]\tr$; iii)
two gear pumps Pump 1 and Pump 2 generating volumetric flows $u_1$ and
$u_2$ respectively such that $u=[u_1,u_2]\tr$; iv) Valve 1 which
splits the flow to Tank 1 ($\gamma_1 u_1$) and Tank 4
($(1-\gamma_1)u_1$); v) Valve 2 which splits the flow to Tank 2
($\gamma_2 u_2$) and Tank 3 ($(1-\gamma_2) u_2$); and vi) Tank 3
drains into Tank 1 whereas Tank 4 drains into Tank 2.  This
cross-coupling creates a system which can be either minimum phase
$1<(\gamma_1+\gamma_2)<2$ or \nonminimum-phase $0<(\gamma_1 +
\gamma_2)<1$.  Using Torricelli's Law the system dynamics for the four
tank process are as follows
\begin{equation}
\label{E:four_tank}
\begin{bmatrix}
\dot{x}_1\\
\dot{x}_2\\
\dot{x}_3\\
\dot{x}_4
\end{bmatrix}=
\begin{bmatrix}
\frac{-a_1 \sqrt{2gx_1} + a_3 \sqrt{2gx_3}}{A_1} \\
\frac{-a_2 \sqrt{2gx_2} + a_4 \sqrt{2gx_4}}{A_2} \\
-\frac{a_3 \sqrt{2gx_3}}{A_3} \\
-\frac{a_4 \sqrt{2gx_4}}{A_4} 
\end{bmatrix}+
\begin{bmatrix}
\frac{\delta_{\gamma 1} \gamma_1}{A_1} & 0\\
0 & \frac{\delta_{\gamma 2} \gamma_2}{A_2} \\
0 & \frac{1 - \delta_{\gamma 2} \gamma_2}{A_3}\\
\frac{1 - \delta_{\gamma 1} \gamma_1}{A_4} & 0
\end{bmatrix} k_u
\begin{bmatrix}
 u_1\\
 u_2
\end{bmatrix}.
\end{equation}  
As was done in \cite{johnsen07:_inter_and_dampin_assig_passiv} we
choose the following integral control states
\begin{equation}
\label{E:four_tank_I}
\begin{bmatrix}
\dot{x}_{\mathsf{I}1}\\
\dot{x}_{\mathsf{I}2}\\
\end{bmatrix} = 
\begin{bmatrix}
k_{\mathsf{I}1} a_1 \sqrt{2 g} \left ( \sqrt{x^*_1} -  \sqrt{x_1}
\right )\\
k_{\mathsf{I}2} a_2 \sqrt{2 g} \left ( \sqrt{x^*_2} -  \sqrt{x_2}
\right )
\end{bmatrix}.
\end{equation}
and corresponding Hamiltonian
\begin{equation}
\label{E:four_tank_ham}
H(x) = \sum_{i=1}^4 \frac{2}{3} k_i a_i \sqrt{2g} x_i^{\frac{3}{2}} +
\sum_{j=1}^2 k_{\mathsf{I}j} a_j \sqrt{x_j^*} x_{\mathsf{I}j}.
\end{equation}
Which results in:
\begin{equation*}
Q_d = \begin{bmatrix}
\frac{-1}{A_1k_1} & 0 & \frac{1}{A_3k_3} & 0 & 0 & 0\\
0 & \frac{-1}{A_2k_2} & 0 & \frac{1}{A_2k_4} & 0 & 0 \\
0 & 0 & \frac{-1}{A_3k_3} & 0  & 0 & 0 \\
0 & 0 & 0 & \frac{-1}{A_3k_3} & 0  & 0 \\
\frac{k_{\mathsf{I}1}}{k_1} & 0 & 0 & 0 & -\frac{k_{\mathsf{I}1}}{k_1}
& 0\\
0 & \frac{k_{\mathsf{I}2}}{k_2} & 0 & 0 & 0 & -\frac{k_{\mathsf{I}2}}{k_2}
\end{bmatrix}
\end{equation*}
which is positive-definite iff 
\begin{gather*}
0 < k_1< \frac{\left ( 4 - k_{\mathsf{I}1} A_1 \right )k_3}{A_3},\ 0 < k_2 < \frac{\left ( 4 - k_{\mathsf{I}2} A_2 \right )k_4}{A_4}\\
0 < k_3,\ k_4 < \infty,\ 0 < k_{\mathsf{I}1} < \frac{4}{A_1},\ 0 < k_{\mathsf{I}2} < \frac{4}{A_2};
\end{gather*}
and the remaining control related terms
\begin{gather*}
P = \begin{bmatrix}
\gamma_1 k_1 & (\gamma_2-1)k_1 \\
(1-\gamma_1)k_2 & \gamma_2 k_2 \\
0 & (1-\gamma_2)k_3\\
(1-\gamma_1)k_4 & 0\\
\gamma_1 k_1 & (1-\gamma_2)k_1 \\
(1-\gamma_1)k_2  & \gamma_2 k_2
\end{bmatrix},\\
\begin{bmatrix}
l_1\\
l_2
\end{bmatrix}= \begin{bmatrix}
-\frac{\gamma_2 a_1\sqrt{2 g x_1^*}+(\gamma_2-1) a_2
  \sqrt{2 g x_2^*}}{\gamma_1+\gamma_2-1}\\
-\frac{(\gamma_1-1) a_1\sqrt{2 g x_1^*}+\gamma_1 a_2
  \sqrt{2 g x_2^*}}{\gamma_1+\gamma_2-1}
\end{bmatrix},\ 
\begin{bmatrix}
x_3^*\\
x_4^*
\end{bmatrix}=
\begin{bmatrix}
\frac{(1-\gamma_2)^2}{2 g a_3^2}l_2^2\\
\frac{(1-\gamma_1)^2}{2ga_4^2}l_1^2
\end{bmatrix}.
\end{gather*}
N.B. it appears that in
\cite{johnsen07:_inter_and_dampin_assig_passiv} the authors
incorrectly applied Sylvester's Criterion directly to $-Q_d$ instead of
the negative of the \Hermitian\ of $Q_d\ (-\hermitian\{Q_d\})$ which
resulted in their clearly incorrect constraints for the controller
gains $k_i$ and $k_{\mathsf{I}}$ for the four tank process.  Finally
our analysis revealed that $x^*_1$ and $x^*_2$ can not be set
completely independent of each other.  Specifically in order for $\grad H_d
(x=x^*,x_{\mathsf{I}}=0,x^*)=0$ then 
\begin{gather*}
\left ( \frac{a_1 \gamma_2}{a_2(1-\gamma_2)} \right )^2 <
\frac{x_2^*}{x_1^*} < \left ( \frac{a_1 (1-\gamma_1)}{a_2\gamma_1}
\right )^2 \text{ if } (\gamma_1 + \gamma_2) < 1\\
\left ( \frac{a_1 (1-\gamma_1)}{a_2 \gamma_1} \right )^2 <
\frac{x_2^*}{x_1^*} < \left ( \frac{a_1 \gamma_2}{a_2(1-\gamma_2)}
\right )^2 \text{ if } (\gamma_1 + \gamma_2) > 1.
\end{gather*}