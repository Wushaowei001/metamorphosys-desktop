Although the presented toolchain in its current early prototype
stage addresses many of the challenges inherent in a unified design
and implementation environment, it lacks many of the desired
features and workflow procedures to be useful in real applications.

The most critical lacking functionality is feedback from the
implementation phases back into the
high-level design. The chosen target platforms enable us to
understand the nature of these influential properties (e.g. jitter,
quantization, constraints on the schedule, fault-tolerance, and computational models). It remains an essential part of
our ongoing quest to widen this set by experimenting with different
deployment platforms. On the other hand, feedback mechanisms will
require formal definitions of such properties; thus, one of the main
focus areas in future development is the formalization of platform
properties and their utilization in the design phases as early as
possible.

A related area in our plans is the design and implementation of an
\emph{adaption layer} on top of the different target architectures.
This layer shall hide many of the differences of the platform APIs
but should be flexible and thin enough not to try to hide the
important characteristics of the underlying platform. The emphasis
is on API adaption, not emulation, and is essential to avoid or to
lessen the need for "glue code" generators. We have already made the
first steps in the design of this layer, e.g. our Linux-based virtual
machine could be considered an early prototype of this attempt.
Ideally, the toolchain should be able to handle any target platform
if the proper adaption layer and a formal set of platform properties
are available.

Finally, the presented tools mostly support a one-way "single shot"
development workflow. Changes to the input MATLAB/Simulink model
require the ECSL-DP model to be regenerated.
Even though this affects only the dataflow aspects of the model,
this step invalidates existing references from the software
component model and deployment models. Clearly, the MATLAB/Simulink
model importer should be prepared to handle existing output models
and should execute the changes only. With properly identifying the
Simulink elements and their corresponding ECSL-DP pairs and by
restricting the set of supported changes to the dataflow models, we will be able to overcome this limitation without bloating
the importer tool.

Along a typical workflow many files are generated and/or modified by
the designer. Most of these files are final output or report files,
but some of them serve as input for later stages in the workflow
process. Currently, the toolchain does not address versioning and
dependency issues between the different tools and intermediate
files. Tool integration in its current form consists of common APIs,
modeling languages, and file formats. The underlying development
tools (GME, UDM, GReAT) provide an excellent framework for
developing an integrated toolset, but this level of integration is
not evident on the user level. We shall provide a high-level
framework---ideally in the form of a GUI application---which executes
the elements of the toolchain, tracks changes and dependencies, and
supports the user in understanding the different stages of the
design process. This approach is very similar to current HDL design
environments \cite{xilinx:ise}\cite{altera:quartus}, which are built
primarily on command line tools with a high-level GUI driver
application.

A related but somewhat more complicated issue with various files and
models is that these files contain related and/or
redundant information. MATLAB/Simulink models have almost the same
information as ECSL-DP dataflow models. Component and deployment
models refer to the dataflow elements and to each other. The
generated code and/or SFC models have clear origins in the ECSL-DP
projects. Because of the different languages and different file
formats, one can find it extremely hard to navigate along these
intricate connections. The high-level integration framework (GUI)
would ideally support the user in this task. Understanding the exact effects (or just the magnitude of
the size of the affected part of the system) and knowing where a
particular information or code artifact is coming from is surely
invaluable in the envisioned design environment.
