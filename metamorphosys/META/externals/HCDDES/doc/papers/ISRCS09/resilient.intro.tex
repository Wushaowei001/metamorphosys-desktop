\section{Introduction}
\label{S:introduction}
%Need an introduction here...
%Gabor, Janos ... ?
%Can we provide other examples for 'health indicators', 'defense optimization'
The design of resilient control systems necessitates novel developments at the 
intersection of computer science and control theory. The control of complex 
dynamic systems is a well-studied area, but much less is known about how to 
implement such control systems that are able to tolerate shortcomings of 
non-ideal software and network-based implementation platforms. Additionally, not 
only implementation side-effects have to be mitigated, but 
also potential issues related to security of the control system. For instance, 
if a network used in the control loop is under a denial-of-service attack, 
we still need to maintain the quality of control for the plant. If the controller
itself is compromised, we need to detach it from a plant and an alternative controller 
must be brought on-line. In this paper we describe a control-theoretical framework based on 
{\em passivity} principles. Passivity-based controllers are ideally suited for 
high-confidence control systems that have infinite gain margins, thus possess a 
great deal of robustness to uncertainty. 

% Key idea:
% Passivity based control principles provide a mathematically rigorous framework to
% construct high-confidence control systems which have infinite gain margins, thus
% possessing a great deal of robustness to uncertainty.  
Passivity is a mathematical property of the controller implementation, and could 
be realized in different ways. The approach described here applies to a large family 
of physical systems  which can be described by both linear and non-linear system models
\cite{haddad08:_nonlin_dynam_system_and_contr,
  schaft99:_l2_gain_passiv_nonlin_contr,
  ortega98:_passiv_based_contr_euler_lagran_system}, including
systems which can be described by cascades of passive systems such as
quad-rotor aircraft \cite{kottenstette08:_digit_passiv_attit_and_altit}.  Furthermore,
the theory can be applied to networked control design
\cite{antsaklis04:_special_issue_networ_contr_system,
  baillieul07:_special_issue} including over wireless networks 
\cite{kottenstette08:_wirel_digit_contr_of_contin}.

For this paper we shall focus on the use of a structure called the
{\em resilient power junction} ( a special type of power junction \cite{kottenstette08:_contr_of_multip_networ_passiv,
  kottenstette09:_digit_contr_of_multip_discr} )
to demonstrate how a passive physical system (in which its dynamics are
described by ordinary differential equations) can be interconnected to
multiple-redundant-passive-digital controllers while maintaining
$L^m_2$-stability.  We shall discuss the conditions for the type of
non-redundant controllers which can be tolerated if no detection
scheme is used.  In addition we demonstrate how 
potentially-destabilizing non-redundant controllers can be removed
from the network when detection of the non-redundant controller
occurs.

Section~\ref{S:wave_variables} reviews wave 
variables with which the power junction interacts with.
Section~\ref{S:ps_ph} reviews the passive sampler and passive hold,
which allow a continuous time plant to be interconnected to a
digital control network.  Section~\ref{S:res_pj} introduces the
resilient power junction (the main contribution to this paper).
Section~\ref{S:stability} provides the main stability result which
shows that $L^m_2$-stability can be maintained in spite of
non-redundant controllers being introduced to the network.
Section~\ref{S:experiments} provides simulated results when various types
of non-redundant controllers are connected to the network.
Section~\ref{S:conclusions} provides conclusions for this paper.