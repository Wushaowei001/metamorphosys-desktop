\section{Conclusions}
\label{S:conclusions}
In this paper we have described how a general technique: passivity,
and a particular controller structure involving the resilient power
junction can be used.  The resilient power junction operating under
Assumption~\ref{A:resilient_pj} when interconnected to
$\mathsf{m_c}-$redundant controllers and a single plant will always
perform well under both denial-of-service attacks on individual
controllers and degrade gracefully as additional \sop\ 'corrupted' digital
controllers are introduced into the network.
However, when introducing a highly-unstable controller into the network
great care must be taken in order to identify and isolate the digital
controller.  Assumption~\ref{A:resilient_pj} had to be made quite a
bit-stricter in order to isolate these unstable controllers, in
particular the time-delays and data-dropouts needed to be identical
when transmitting controller wave variables to and from the
power-junction.  This can be fairly easily satisfied on a
real-time-operating system but more difficult over a network.  We did
provide the important result, however, that controllers can be removed
without either destabilizing the system and showed that they can still
maintain uninterrupted performance.

The theoretical framework presented gives a tool to the control
engineer for building digital control systems that can survive, and
even 'operate through' attacks, while maintaining the quality of
control.  Naturally, there are critical points in the implementation
(e.g. the realization of the resilient power junction) that needs to
be created  with great care.  In any case, passivity-based approaches
to controller design provide a promising direction for designing
controllers that are significantly more robust than other techniques.
As illustrated, mathematical proofs exist for their properties, and
they could be widely applied to linear and non-linear systems alike.
