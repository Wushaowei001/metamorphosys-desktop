\section{BIP Design and Analysis}

%%%%%%%%%%%%%%%%%%%%%%%%%%%%%%%%%%%%%%%%%%%%%%%%%%%%%%%%%%%%%%%%%%%%%%%%
%%%%%%%%%%%%%%%%%%%%%%%%Abstract syntax of BIP --- Component based construction%%%%%%%%%%%%%%%%%%%%
%%%%%%%%%%%%%%%%%%%%%%%%%%%%%%%%%%%%%%%%%%%%%%%%%%%%%%%%%%%%%%%%%%%%%%%%
\subsection{Component based construction}
A \textbf{port} $p$ is defined by the port identifier $p$, and the data variables associated with the port. \newline 
An \textbf{atomic component} is a Petri net extended with data. It consists of a set of ports $P$ used for the synchronization with other components, a set of transitions $T$ and a set of local variables $X$. Transitions describe the behaviour of the component. They are represented as a labelled relation on the set of control locations $L$.

Figure \ref{atom} shows an example of an atomic component with two ports $r_1$, $t_1$, a variable $x$, and two control locations $l_1$, $l_2$. At control location $l_1$, the transition labelled $t_1$ is possible. When an interaction through $t_1$ takes place, a random value is assigned for the variable $x$. This value is exported through the port $r_1$. From the control location $l_2$, the transition labelled $r_1$ can occur (the guard of the transition is true by default), the variable $x$ is eventually modified and the value of $x$ is printed.
\begin{figure}[htbp]
  \begin{center}
        \input{images/atom.pdf_t}
 \end{center}
  \caption{An example of an atomic component in BIP} \label{atom}
\end{figure}


An \textbf{interaction} $a$ is defined by 1) $P$ a support set of ports 2) $G$ a guard, which is an arbitrary predicate on the variables of its ports, 3) $U$ an upward update function 4) $D$ a downward update function.\newline
Synchronization through an interaction involves three steps: 1) The computation of the upward update function $U$ ; 2) Evaluation of the guard $G$ 3) The computation of the downward update function $D$. we note that we can define hierarchical interactions \cite{verif:bs08,verif:bs09}, however, we can flatten hierarchy~\cite{verif:bjs09}, and hence, we just consider a set of non-hierarchical interactions. \newline
One can add \textbf{priorities} to reduce non-determinism whenever many interactions are enabled. Then the interaction with the higher priority would be chosen. These priorities can also be guarded.\newline
A \textbf{Composite component} $C$ is defined from existing atomic, and from a set of interactions which are connected to the atomic components. We note that a composite component obtained by composition of a set of atomic components can be composed with other components in a hierarchical and incremental fashion using the same operational semantics. However, it is also possible to flatten a composite component and obtain a non-hierarchical one \cite{verif:bjs09}.




%%%%%%%%%%%%%%%%%%%%%%%%%%%%%%%%%%%%%%%%%%%%%%%%%%%%%%%%%%%%%%%%%%%%%%%%
%%%%%%%%%%%%%%%%%%%%%%%%Semantics of execution of BIP%%%%%%%%%%%%%%%%%%%%
%%%%%%%%%%%%%%%%%%%%%%%%%%%%%%%%%%%%%%%%%%%%%%%%%%%%%%%%%%%%%%%%%%%%%%%%

\subsection{Engine Protocol}  \label{sec:enumprotocol}
The operational semantics is implemented by the BIP engine. In the basic implementation, 
the engine computes the enabled interactions by enumerating the complete list of possible 
interactions in the model. At each step, the engine selects the enabled interactions from 
the complete list of interactions, based on the current state of the atomic components. 
Then, among the enabled interactions, priority rules are applied to eliminate the ones with 
lower priority.\\
The main loop of the engine consists of the following steps:
\begin{enumerate}
\item Each atomic component notifies the engine of its current state.
\item The engine enumerates the allowed interactions in the model and selects the enabled ones based on the current state of the atomic components.
\item Selected interactions are filtered according to the priority model to keep only those with highest priority.
\item The engine then selects one of the remaining interactions, and notifies the involved atomic components of their respective transitions to take.
\end{enumerate}

 
\subsection{Send/Receive}
Why and how ? To do .................................
