\section{Background}
\label{section:background}

%\subsection{ESMoL Component Model}

A number of frameworks and techniques contributed to our solution:

\textbf{ESMoL Component Model}: As the ESMoL language structure is documented elsewhere\cite{modeling:esmol_tr}, we only cover details relevant to incremental cycle checking.  ESMoL is a graphical
modeling language which allows designers to use Simulink diagrams as 
synchronous software function specifications (where the execution of each 
block is equivalent to a single bounded-time blocking C language call).  These
specifications are used to create blocks representing ESMoL component types.   
ESMoL components have message structures as interfaces, and the type
specification includes a map between Simulink signal ports and the 
fields of the input and output message structures.  The messages represent C structures, and the map graphically captures the marshaling of Simulink data to those structures.  ESMoL and its tools provide design concepts and functions for specifying  logical architecture and deployment, and performing scheduling analysis.

In ESMoL a designer can include Simulink references from
parts of an imported dataflow model, and instantiate them any number of 
times within the type definitions.  ESMoL tasks can distribute functions over a time-triggered network for performance, or replicate similar functions for fault mitigation.  
This level of flexibility requires automatic type-checking to ensure 
compatibility for chosen configurations.  Beyond interface type-checking, 
structural well-formedness problems arise during assembly such as zero-delay 
cycles.

%\subsection{Cycle Enumeration}

\textbf{Cycle Enumeration}: To implement cycle enumeration we use the algorithm Johnson proposed as an extension of Tiernan's algorithm \cite{cycles:tiernan} for enumerating elementary cycles in a directed graph\cite{cycles:johnson75}.  Both approaches rely on depth-first search with backtracking, but Johnson's method marks vertices on elementary paths already considered to eliminate fruitless searching, unmarking them when a cycle is found.  Johnson's algorithm is polynomial ($O((n + e)c)$, where $n$, $e$, and $c$ are the sizes of the vertex, edge, and cycle set, respectively), and is considered the best available general cycle enumeration method\cite{cycles:mateti76}.  We implemented Johnson's algorithm in C++ using the Boost Graph library \cite{tools:boostgraph}.

%\subsection{Hierarchical Graphs}

\textbf{Hierarchical Graphs}: As a notation for describing our incremental approach we use the algebra of hierarchical graphs introduced by Bruni et al\cite{graphs:hier_algebra}. We will only give a summary of some of the notation here for brevity.  The interested reader can refer to \cite{modeling:esmol_cycles_tr} (and \cite{graphs:hier_algebra}) for a more detailed account.

\begin{align}
\label{eq:hiergraphs}
\mathbb{D} &::= L_{\bar{x}} [ \mathbb{G} ]  \\
\mathbb{G} &::= \mathbf{0} \mid x \mid l<\bar{x}> \mid \mathbb{G} \parallel \mathbb{G} \mid  (\nu \bar{x})\mathbb{G} \mid \mathbb{D} <\bar{x}>  \nonumber
\end{align}

Intuitively, Equation \ref{eq:hiergraphs} is a grammar defining a simple textual notation for describing typed hierarchical graphs.  Within the formalism we can compare equivalence between algebraic descriptions of two hierarchical graphs using reduction rules and a normal form (as in Bruni\cite{graphs:hier_algebra}), though equivalence is beyond the scope of this publication. The algebraic properties are for future use. The other main attraction of this particular formalism is that the notation allows the definition of interface symbols which correspond easily to port objects in a dataflow language, and the hiding of those interfaces as we specialize types.  The notation is a compact shorthand for much larger diagrams or mathematical descriptions. The rule $\mathbb{D}$ corresponds to composite types in our dataflow language (which may have other composites as children).   The specification for a composite element is $L_{\bar{x}} [\mathbb{G} ]$, which means that an element of $\mathbb{D}$ has type $L$ and interface vertices in the list $\bar{x}$ and a corresponding internal graph $\mathbb{G}$ defining the details of the component. The internal graph may also include subcomponents. Gluing of subgraphs  only occurs at common vertices.  When a composite element from $\mathbb{D}$ of a particular type is used as a child element to form a  larger (parent) graph, vertices from the child are possibly renamed in the parent, hence the notation $\mathbb{D} <\bar{x}> $.  In a parallel composition, vertices with the same name $x$ are glued together.   Finally $\llbracket \mathbb{G} \rrbracket$ indicates the graph corresponding to the expression $\mathbb{G}$.

