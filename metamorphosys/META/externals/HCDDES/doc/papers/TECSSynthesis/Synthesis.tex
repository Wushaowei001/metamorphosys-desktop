\section{Synthesis of Code and Analysis Models}

\subsection{Functional Code}

\cite{mic:ecsldp} describes a much earlier version of the modeling language and
tools created to support automotive software development.  The Simulink model
importer and graph transformation-based code generation for the control
functions are documented in \cite{mic:ecsldp}.  The tools still support the
importing of Simulink functions, and generation from Simulink subsystem blocks
to synchronous C code.  The main advantage over the Simulink Real-Time Workshop
code generator \cite{tools:mathworks} is the easy customizability of our
generator.  Generation is based on simple code templates for each functional
block which can be created new or customized by the user.  Transformation details
are also open to inspection and modification.

\subsection{Scheduler}

The scheduling tool is a good example of the second-stage generation
process.  We will consider one example of an analysis artifact.  For generation other than the
functional code, we use the CTemplate engine\cite{tools:ctemplate} called from C++
code.

The host section of the template appears below, followed by an example
generated from the quad integrator model:

\newpage

\begin{alltt}
\scriptsize
\{\{\#HOST_SECTION\}\} Proc \{\{NODENAME\}\}\{\{NODEFREQ\}\}\{\{SENDOHD\}\}\{\{RECVOHD\}\}
\{\{\#TASK_SECTION\}\}Comp \{\{TASKNAME\}\} = \{\{FREQUENCY\}\}\{\{WCEXECTIME\}\}
\{\{\/TASK_SECTION\}\}\{\{\#LOCAL_MSG_SECTION\}\}Msg \{\{MSGNAME\}\} \{\{MSGSIZE\}\} \{\{SENDTASK\}\}\{\{RECVTASKS\}\}
\{\{\/LOCAL_MSG_SECTION\}\}
\{\{/HOST_SECTION\}\}

---------------------------------------------------------------------

Proc RS 4MHz 0s 0s
Comp InnerLoop =50Hz 1.9ms
Comp DataHandling =50Hz 1.8ms
Comp SerialIn =50Hz 1us
Comp SerialOut =50Hz 1ms
Msg DataHandling.sensor_data_in 1B RS/SerialIn RS/DataHandling 
Msg InnerLoop.thrust_commands 37B RS/InnerLoop RS/SerialOut
Msg DataHandling.ang_msg 1B RS/DataHandling RS/InnerLoop 

Proc GS 100MHz 0s 0s
Comp RefHandling =50Hz 1us
Comp OuterLoop =50Hz 245us
Msg RefHandling.pos_ref_out 9B GS/RefHandling GS/OuterLoop
\end{alltt}

In CTemplate, each \verb${{#...}} {{/...}}$ tag pair delimits a section which
can be repeated by filling in the proper data structure in the code.  The other
tags \verb${{...}}$ are replaced by the string specified in the generation
code. Our model has two processors (Proc lines, above), each of which runs
multiple components (Comp) and has at least one local data dependency (Msg).

Producing the Proc and Comp lines from the model API is straightforward,
as these are simple traversals of the model, respecting the model hierarchy. 
Each generated line uses parameters from the respective model object. 
The parameters are shown only in the generated output, though the object diagram in 
Fig. \ref{fig:msg_sched} illustrates a good example of parameter layout and disposition. 
In order to produce each Msg line, many relations must be collected (as shown 
in table \ref{tab:localdeps}) and distilled into the right relationships.  
This requires complex traversal code.  To write a generator similar to this one, the developer uses the 
interpreter API and the transformed abstract syntax graph model.  In the abstract language
traversal we collect the LocalDependency objects and filter them by processor. 
Each LocalDependency object contains all of the information necessary to fill
out the parameters in the template and create a new Msg line in the
configuration file.

\subsection{Platform-Specific Code}

The template description for the platform-specific code generator is shown here:

\begin{alltt}
\scriptsize
////////////////////////////////// SCHEDULE /////////////////////////////////

portTickType hp_len = \{\{NODE_hyperperiod\}\};

\{\{\#SCHEDULE_SECTION\}\}
task\_entry tasks[] = \{
\{\{\#TASK\}\}
   \{ \{\{TASK_name\}\}, "\{\{TASK_name\}\}", \{\{TASK\_startTime\}\}, 0\},\{\{\/TASK\}\}
   \{NULL, NULL, 0, 0\}
\}\;

msg\_entry msgs[] = \{
\{\{\#MESSAGE\_NAME\}\}
   \{ \{\{MESSAGE\_index\}\}, \{\{MESSAGE\_sendreceive\}\}, sizeof( \{\{MESSAGE\_name\}\} ), 
         (portCHAR *) \& \{\{MESSAGE\_name\}\}, 
         (portCHAR *) \& \{\{MESSAGE\_name\}\}\_c, \{\{MESSAGE\_startTime\}\}, 
         pdFALSE\},
\{\{\/MESSAGE\_NAME\}\}
   \{ -1, 0, 0, NULL, NULL, 0, 0\}
\}\;

per\_entry pers[] = \{
\{\{\#PER_NAME\}\}
   \{ \{\{PER\_index\}\}, "\{\{PER\_type\}\}", 
         \{\{PER\_way\}\}, 0, \{\{PER\_pin\_number\}\}, sizeof( \{\{PER\_name\}\} ), 
         (portCHAR *) \& \{\{PER\_name\}\}, 
         (portCHAR *) \& \{\{PER\_name\}\}\_c, 
         \{\{PER\_startTime\}\}, NULL\}, 
\{\{\/PER_NAME\}\}
   \{ -1, NULL, 0, 0, 0, 0, NULL, NULL, 0, NULL \}
\}\;
\{\{\/SCHEDULE_SECTION\}\}

-----------------------------------------------------------------------------

////////////////////////////////// SCHEDULE /////////////////////////////////

portTickType hp_len = 20;

task\_entry tasks[] = \{
   \{ DataHandling, "DataHandling", 4, 0\},
   \{ InnerLoop, "InnerLoop", 9, 0\},
   \{NULL, NULL, 0, 0\}
\}\;

msg\_entry msgs[] = \{
   \{ 1, MSG\_DIR\_RECV, sizeof( OuterLoop\_ang\_ref ), 
      (portCHAR *) \& OuterLoop\_ang\_ref, 
      (portCHAR *) OuterLoop_ang\_ref\_c, 7, 0, 0\},
   \{ 2, MSG\_DIR\_SEND, sizeof( DataHandling\_pos\_msg ), 
      (portCHAR *) \& DataHandling\_pos\_msg, 
      (portCHAR *) DataHandling\_pos\_msg\_c, 11, 0, 0\},
   \{ -1, 0, 0, NULL, NULL, 0, 0, 0\}
\}\;

per\_entry pers[] = \{
   \{ 1, "UART", IN, 0, 0, sizeof( DataHandling\_sensor\_data\_in ), 
      (portCHAR *) \& DataHandling\_sensor\_data\_in, 
      (portCHAR *) DataHandling\_sensor\_data\_in\_c, 2, NULL, 0, 0\},
   \{ 2, "UART", OUT, 0, 0, sizeof( InnerLoop\_thrust\_commands ), 
      (portCHAR *) \& InnerLoop\_thrust\_commands, 
      (portCHAR *) InnerLoop\_thrust\_commands\_c, 14, NULL, 0, 0\},
   \{ -1, NULL, 0, 0, 0, 0, NULL, NULL, 0, NULL, 0, 0 \}
\}\;

\end{alltt}

The FRODO virtual machine generation template brings together all of the
semantic relations described in the earlier section.  The template and generated code
segment above correspond to the second-stage interpreter that creates the static schedule
structures used by the virtual machine.  The tasks, messages, and peripherals
listed here come from the Acquisition, Actuation, Transmit, and Receive
relation objects.  The various connected objects are sorted according to
schedule time, and then the template instantiation uses the object parameters
to create the tables in a manner similar to that described for the scheduler
configuration generation above.  For the curious, the LocalDependency relations
do not appear in this table.  The scheduler creates constraints that must be
satisfied for each local dependency.  Therefore, any valid task and message
schedule will satisfy them.  In a different part of the FRODO template, the
local dependencies determine which message fields must be used as
arguments to the component function calls (not shown).