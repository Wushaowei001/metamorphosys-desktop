\section{Future Work: Deadlock Analysis Using BIP}

Deadlock analysis of a model in our tools requires consideration of both the control system 
components and the inter-component communications controlled from within the virtual machine.  
Analysis of the control system components (within tasks) is simplified as we assume a 
synchronous dataflow execution paradigm. This guarantees deadlock free local execution as 
long as the overall network is schedulable \cite{moc:sdf}.  Of greater interest is deadlock 
analysis of the communications controllers and virtual machine execution when the control 
components have passed a message off for transfer.

The BIP \cite{verif:basu, verif:BIPFramework} tool chain includes facilities for performing 
deadlock analysis on componentized models.  BIP (which stands for Behavior, Interaction, 
Priority) is built upon the labeled transition system formalism for component behaviors, 
allowing model checking with the IF Toolset  \cite{verif:IFToolset}. Interactions are used 
to compose components and are able to maintain properties such as deadlock freedom, through 
composition.  Priorities are applied between interactions to select from a group of enabled 
transitions.

The structure of the inter-component communications controllers and virtual machine do not 
vary from one model to another -- their execution behavior is only altered by the schedule 
data.  Once a valid schedule has been generated, the intermediate model contains all of the 
information necessary to translate into an equivalent BIP model.  A model interpreter will 
generate the BIP output from the semantic model using templatized BIP-component models 
substituting scheduling information where necessary and creating interactions and priorities 
between components as appropriate. Significant initial effort must go into creating the 
templatized component models to accurately reflect the execution semantics of the communications 
controllers and virtual machine.  This effort is the foundation upon which our BIP integration 
is based.

While we are in the early design stages of BIP integration, it already promises to be able to 
faithfully capture the semantics of our models.  The LTS paradigm is a natural adaptation from 
the approach taken in the design of much of the virtual machine, and the available interaction 
structures are rich enough to represent our distributed environment. Additional parameters may 
need to be added to the component templates if other types of analysis beyond deadlock are required.
